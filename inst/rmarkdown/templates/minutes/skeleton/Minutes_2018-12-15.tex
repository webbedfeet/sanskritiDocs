\documentclass[12pt,]{article}
\usepackage{lmodern}
\usepackage{amssymb,amsmath}
\usepackage{ifxetex,ifluatex}
\usepackage{fixltx2e} % provides \textsubscript
\ifnum 0\ifxetex 1\fi\ifluatex 1\fi=0 % if pdftex
  \usepackage[T1]{fontenc}
  \usepackage[utf8]{inputenc}
\else % if luatex or xelatex
  \ifxetex
    \usepackage{mathspec}
  \else
    \usepackage{fontspec}
  \fi
  \defaultfontfeatures{Ligatures=TeX,Scale=MatchLowercase}
    \setmainfont[]{Comic Sans MS}
\fi
% use upquote if available, for straight quotes in verbatim environments
\IfFileExists{upquote.sty}{\usepackage{upquote}}{}
% use microtype if available
\IfFileExists{microtype.sty}{%
\usepackage{microtype}
\UseMicrotypeSet[protrusion]{basicmath} % disable protrusion for tt fonts
}{}
\usepackage[margin=0.75in, letterpaper, head = 2cm, includehead]{geometry}
\usepackage{hyperref}
\hypersetup{unicode=true,
            pdftitle={Sanskriti BOD 2019 Minutes},
            pdfauthor={Abhijit Dasgupta},
            pdfborder={0 0 0},
            breaklinks=true}
\urlstyle{same}  % don't use monospace font for urls
\usepackage{longtable,booktabs}
\usepackage{graphicx,grffile}
\makeatletter
\def\maxwidth{\ifdim\Gin@nat@width>\linewidth\linewidth\else\Gin@nat@width\fi}
\def\maxheight{\ifdim\Gin@nat@height>\textheight\textheight\else\Gin@nat@height\fi}
\makeatother
% Scale images if necessary, so that they will not overflow the page
% margins by default, and it is still possible to overwrite the defaults
% using explicit options in \includegraphics[width, height, ...]{}
\setkeys{Gin}{width=\maxwidth,height=\maxheight,keepaspectratio}
\IfFileExists{parskip.sty}{%
\usepackage{parskip}
}{% else
\setlength{\parindent}{0pt}
\setlength{\parskip}{6pt plus 2pt minus 1pt}
}
\setlength{\emergencystretch}{3em}  % prevent overfull lines
\providecommand{\tightlist}{%
  \setlength{\itemsep}{0pt}\setlength{\parskip}{0pt}}
\setcounter{secnumdepth}{0}
% Redefines (sub)paragraphs to behave more like sections
\ifx\paragraph\undefined\else
\let\oldparagraph\paragraph
\renewcommand{\paragraph}[1]{\oldparagraph{#1}\mbox{}}
\fi
\ifx\subparagraph\undefined\else
\let\oldsubparagraph\subparagraph
\renewcommand{\subparagraph}[1]{\oldsubparagraph{#1}\mbox{}}
\fi

%%% Use protect on footnotes to avoid problems with footnotes in titles
\let\rmarkdownfootnote\footnote%
\def\footnote{\protect\rmarkdownfootnote}

%%% Change title format to be more compact
\usepackage{titling}

% Create subtitle command for use in maketitle
\newcommand{\subtitle}[1]{
  \posttitle{
    \begin{center}\large#1\end{center}
    }
}

\setlength{\droptitle}{-2em}

  \title{Sanskriti BOD 2019 Minutes}
    \pretitle{\vspace{\droptitle}\centering\huge}
  \posttitle{\par}
    \author{Abhijit Dasgupta}
    \preauthor{\centering\large\emph}
  \postauthor{\par}
      \predate{\centering\large\emph}
  \postdate{\par}
    \date{December 15, 2018}

\usepackage{graphicx}
\usepackage{fancyhdr}
\fancypagestyle{style1}{
\rhead{\includegraphics[width=0.9in]{logo}}
\lhead{\Large Sanskriti BOD 2019 Minutes}
}
\makeatletter
\renewcommand{\maketitle}{\bgroup\setlength{\parindent}{0pt}
\begin{flushleft}
  %\textbf{\@title}
  \textbf{Prepared by:} \@author\\
  \textbf{Meeting date:} \@date
\end{flushleft}\egroup
}
\makeatother
\pagestyle{style1}

\begin{document}
\maketitle

%\maketitle
\thispagestyle{style1}

{
\setcounter{tocdepth}{4}
\tableofcontents
}
\hypertarget{attendees}{%
\subsection{Attendees}\label{attendees}}

\begin{longtable}[]{@{}ll@{}}
\toprule
\endhead
\begin{minipage}[t]{0.09\columnwidth}\raggedright
\textbf{Present:}\strut
\end{minipage} & \begin{minipage}[t]{0.85\columnwidth}\raggedright
Jayati Bera, Abhijit Dasgupta, Lina Chakraborty, Arunima Ghosh,
Chiranjib Sanyal, Swapan Sinha, Nabanita Nag\strut
\end{minipage}\tabularnewline
\begin{minipage}[t]{0.09\columnwidth}\raggedright
\textbf{Absent:}\strut
\end{minipage} & \begin{minipage}[t]{0.85\columnwidth}\raggedright
Sanjoy Bose, Soumyadeep Dey\strut
\end{minipage}\tabularnewline
\bottomrule
\end{longtable}

\newpage

\hypertarget{puja-schedule}{%
\subsection{Puja Schedule}\label{puja-schedule}}

The proposed schedule for Saraswati Puja was circulated:

\begin{longtable}[]{@{}lll@{}}
\toprule
Day & Event & Time\tabularnewline
\midrule
\endhead
Friday & Decoration \& kitchen setup & 8:00 - 11:00 AM\tabularnewline
& &\tabularnewline
Saturday & Puja Prep & 10:00 AM\tabularnewline
& Puja/Hate-khodi & 11:00AM -- 1:00 PM\tabularnewline
& Prasad \& Lunch & 1:00 to 2:30 PM\tabularnewline
& Recitation \& Drawing competition & 1:00 to 4:00 PM\tabularnewline
& Cultural Program & 4:00 to 8:00 PM\tabularnewline
& Dinner & 8:00 to 9:30 PM\tabularnewline
& Clean-up & till 11:00 PM\tabularnewline
\bottomrule
\end{longtable}

It was felt that starting setup at 8pm on Friday was too late for a
winter evening, and that we needed to start lunch earlier to accommodate
an early start to the children's competitions, which in turn would allow
an earlier start to the cultural program. The following modifications
were proposed and approved by the board:

\begin{longtable}[]{@{}ll@{}}
\toprule
Event & Modified Time\tabularnewline
\midrule
\endhead
Friday set-up & 7:00 - 11:00 pm\tabularnewline
Lunch & 12:30 - 2:30 pm\tabularnewline
Kid's program & 1:00 - 3:00 pm\tabularnewline
Cultural program & 3:00 - 8:00 pm\tabularnewline
\bottomrule
\end{longtable}

Ms.~Ghosh informed the Board that the audio engineer had requested 90
minutes for setup before the first program. There was concern that there
was insufficient buffer time at the beginning to allow proper audio
setup, so it was proposed that we rent the auditorium an hour earlier to
ensure a timely start to the cultural program at 3:00 pm.

\hypertarget{action-items}{%
\paragraph{Action items}\label{action-items}}

\begin{enumerate}
\def\labelenumi{\arabic{enumi}.}
\tightlist
\item
  The \textbf{President} will request Montgomery County (Ms.~King) to
  extend the rental so we have access to the cafeteria and kitchen at 7
  pm on Friday, February 8 and the auditorium at 1:00 PM on Saturday,
  February 9.
\item
  The Board will prepare the first announcement for Saraswati Puja, to
  be distributed December 26. This will also announce the opening of
  online registration.
\item
  Mr.~Dasgupta will prepare the online registration system reflecting
  the rates decided (see below)
\item
  Mr.~Dasgupta will post the Saraswati Puja ``Save the date'' and will
  follow up with the announcement on December 26.
\end{enumerate}

\hypertarget{puja}{%
\subsection{Puja}\label{puja}}

Ms.~Bera informed us that Mr.~Bandy was very resistant to continuing his
service as the main priest for Puja in 2019, stating concerns about his
advancing age and having to drive at night. She proposed several
potential alternatives, including Somesh Chattopadhay, Bidhan
Chakraborty, Biswajit Chatterjee and Nilotpal Kundagrami. There was also
an option to have one of the above perform Puja while Mrs Aloka
Chakravarty recited the \emph{shlokas}, since there was positive
feedback from Durga Puja 2018 about her pronounciation and abilities.
Given the short time remaining till Saraswati Puja, it was decided that
Ms.~Bera would call Mr.~Bandy again to request that he perform Saraswati
Puja for this year, allowing us to decide on replacement(s) for Durga
Puja, buying us a bit of time for this important role.

The issue of saris used during Puja and their distribution was raised.
Ms.~Bera reminded us that in the past, saris offered for puja were kept
and given as prizes during the year, for example, during the picnic.
This year, saris were distributed by Mr.~Bandy to the ladies present and
helping during Puja, but not to all the helpers or in any systematic
manner, leading to some discontent. She also reminded us that in the
past, Sanskriti had provided good quality saris for puja, and also given
sweaters to Mr.~Bandy and his helper(s) as a token of appreciation. She
proposed that we authorize Mr.~Manik Ghosh, who will be visiting Kolkata
this winter, to purchase 2 saris for Puja, with a budget of Rs. 2500 for
each sari, to ensure a reasonable quality is purchased. This translates
to a budget item of \$100 for sari purchase.

\hypertarget{action-items-1}{%
\paragraph{Action items}\label{action-items-1}}

\begin{enumerate}
\def\labelenumi{\arabic{enumi}.}
\tightlist
\item
  The President will call Mr.~Bandy again to request he perform priest
  duties this Saraswati Puja. She will inform the Board of the results
  of this conversation.
\end{enumerate}

\hypertarget{food}{%
\subsection{Food}\label{food}}

Mr.~Sanyal reassured the Board that he is willing and available to
continue leading the food efforts during Puja this year. Mr.~Sinha will
be assisting him in this area.

Mr.~Sinha proposed that we might want to have some extra trays of food
on hand since members liked to take home any available extra food
\emph{(this was my understanding)}

The menu for Saraswati Puja was decided, mainly in line with the 2018
menu.

\textbf{Friday}

\begin{longtable}[]{@{}ll@{}}
\toprule
Adults & Kids\tabularnewline
\midrule
\endhead
Chanachur & Cheese Pizza\tabularnewline
Muri &\tabularnewline
Bonde &\tabularnewline
\bottomrule
\end{longtable}

\textbf{Saturday}

\begin{longtable}[]{@{}lll@{}}
\toprule
Meal & Adults & Kids\tabularnewline
\midrule
\endhead
Lunch & Khichuri & Pizza\tabularnewline
& Labra & Juice\tabularnewline
& Veg cutlet & Chips\tabularnewline
& Beguni/aloo bhaja & Cupcake\tabularnewline
& &\tabularnewline
Snack & Muri & Chicken nuggets\tabularnewline
& Chanachur & Juice\tabularnewline
& Singara &\tabularnewline
& &\tabularnewline
Dinner & Peas pulao & Chicken noodles\tabularnewline
& Chholar daal & Grilled chicken piece\tabularnewline
& Goat curry & Brownie bites\tabularnewline
& Cabbage with peas &\tabularnewline
& Dhokar dalna / malai kofta &\tabularnewline
& (missing dessert) &\tabularnewline
\bottomrule
\end{longtable}

It was decided that this contract would go to Spice Grill given their
excellent performance in 2018. Singara may be procured from Ritu Sharma.

\hypertarget{decoration}{%
\subsection{Decoration}\label{decoration}}

Mr.~Haripada Sarker will continue to lead the decoration efforts.
Mr.~Dasgupta will work with Mr.~Sarker to recruit young volunteers to
help him.

Mr.~Anutosh Saha, who has kindly stored all puja decorations in his home
for several years, intimated to Mr.~Ghosh (past-president) that he would
like the decorations moved so he could carry out some home renovations.
It was felt that despite the increased costs, it would be best to obtain
a larger storage to store all of Sanskriti's materials, and that
well-wishers may possibly offset the increased cost. However, since it
would be difficult to get a storage in this short time, Ms.~Bera would
request Mr.~Saha to let us store the materials till Durga Puja. At that
point, a new storage facility will be rented and all the materials,
which will be used at Durga Puja anyway, could be stored in the new
location after Puja.

\hypertarget{action-items-2}{%
\paragraph{Action items}\label{action-items-2}}

\begin{enumerate}
\def\labelenumi{\arabic{enumi}.}
\tightlist
\item
  Ms Bera, and Mssrs. Dasgupta and Sanyal will create a list of
  potential storage locations and call them to find out about prices.
  Mr.~Dasgupta will contact the current storage facility to find out
  about the size of the current storage.
\item
  Ms.~Bera will call Mr.~Saha to discuss this arrangement and whether he
  is amenable to it. She will report back to the Board.
\end{enumerate}

\hypertarget{cultural-program}{%
\subsection{Cultural Program}\label{cultural-program}}

Ms.~Bera reminded us of the rules and principles to be followed in
deciding upon acts who will perform at Saraswati Puja:

\begin{itemize}
\tightlist
\item
  Support new groups who have not performed recently
\item
  Large groups get priority
\item
  All participants must register for Puja
\end{itemize}

She also stated that the final acts as decided by the cultural secretary
be discussed in board.

Ms.~Ghosh informed the Board that the deadline for applications was
midnight on the 15th, so applications were still coming in. Thus far, 3
dance groups, 2 drama troupes (Cultural Creation and Anubrata Choudhury)
and several music acts had applied. From local artists, there would be 3
dance performances, one each by elementary, middle school and high
school kids, directed by Ms.~Priyanka Das, Ms.~Arpita Sabud and
Ms.~Swati Choudhury, respectively. Both drama applications were highly
regarded as well. There was an application for an opening musical act
featuring Shardul Ghosh on sitar and Abjini Chattopadhyay on piano. Both
are talented young artists. In terms of paid guest artists, there were
three contenders:

\begin{enumerate}
\def\labelenumi{\arabic{enumi}.}
\tightlist
\item
  Shekhar Das, playing Electric Guitar (30 mins, \$300)
\item
  Diptanu Das accompanied by a male lead and a small band, singing
  Bangla + Hindi songs (\$1000)
\item
  Biplab, also a singing act
\end{enumerate}

After a robust discussion of the composition of the portfolio of acts
performing at Saraswati Puja and listening to recordings of the
potential guest artists, it was decided that both Mr.~Shekhar Das and
Ms.~Diptanu Das would be approached to perform, and that their charges
would be negotiated, with a starting position of \$250 for Mr.~Shekhar
Das and \$700 for Ms.~Diptanu Das.

The timings for the cultural program will be as follows:

\begin{longtable}[]{@{}ll@{}}
\toprule
Time & Event\tabularnewline
\midrule
\endhead
3:00-5:30 & Program\tabularnewline
5:30 - 6:30 & Break (sound test for guest artiste)\tabularnewline
6:30 - 6:45 & Prize distribution for children\tabularnewline
6:30 - 8:00 & Program\tabularnewline
\bottomrule
\end{longtable}

\textbf{Kid's competitions}

Mr.~Sinha will lead the kid's competitions like last year. Both the
drawing and recitation competitions will be held in 3 age groups: 6-8,
9-11, 12-14. The drawing competition will be held first in the interests
of time, since many participating children will also be in the cultural
programs in the evening. Mr.~Sinha solicited suggestions for poems and
drawing topics from the Board.

\hypertarget{action-items-3}{%
\paragraph{Action items}\label{action-items-3}}

\begin{enumerate}
\def\labelenumi{\arabic{enumi}.}
\tightlist
\item
  Ms.~Ghosh and Ms.~Bera will negotiate with Mr.~Shekhar Das and
  Ms.~Diptanu Das regarding renumeration for performing at Saraswati
  Puja, and will report back to the Board
\item
  Mr.~Sinha will circulate drawing topics and poems to the Board, and
  these will be finalized by December 26, when the first Saraswati Puja
  newsletter will be distributed
\item
  Mr.~Sinha will work on finding judges for the various competitions as
  well as a senior member of the community who will distribute prizes.
  He solicited the Board's help in this.
\end{enumerate}

\hypertarget{subscription}{%
\subsection{Subscription}\label{subscription}}

It was decided to keep the same subscription rates as last year, with
the modification that the family rate would include 2 children aged 6-18
years.

\begin{longtable}[]{@{}llll@{}}
\toprule
Member & Early bird special (12/26-1/19) & Online (1/20-2/3) & At the
gate\tabularnewline
\midrule
\endhead
Individual & \$35 & \$40 & \$45\tabularnewline
Couple & \$65 & \$75 & \$85\tabularnewline
Student & \$20 & \$25 & \$30\tabularnewline
Child/Dependent Parent & \$20 & \$25 & \$25\tabularnewline
Family (2 adults + 2 children 6-18 years) & \$105 & \$120 &
\$130\tabularnewline
Child (5 and below) & Free & Free & Free\tabularnewline
\bottomrule
\end{longtable}

Vendors will be charged \$150 and we will entertain a maximum of 10
vendors

\hypertarget{rental-agreement}{%
\subsection{Rental agreement}\label{rental-agreement}}

The rental agreement was received, along with a receipt of payment.

\hypertarget{action-items-4}{%
\paragraph{Action items}\label{action-items-4}}

\begin{enumerate}
\def\labelenumi{\arabic{enumi}.}
\tightlist
\item
  Ms.~Bera will request additional rentals of 1 hour for the cafeteria
  on Friday and the auditorium on Saturday. Ms.~Chakraborty will pay the
  additional fee when the invoice is received.
\item
  Ms.~Bera, and Mssrs. Dasgupta and Bose will take the Facility
  Training.
\end{enumerate}

\hypertarget{newsletter}{%
\subsection{Newsletter}\label{newsletter}}

It was decided to revive the annual newsletter. Mssrs. Dasgupta and Bose
will lead this. The dates for the Sanskriti events for 2019 were
tentatively fixed:

\begin{longtable}[]{@{}lll@{}}
\toprule
Event & Date & Location\tabularnewline
\midrule
\endhead
Baisakhi & April 20, 2019 & County middle school\tabularnewline
Picnic & Sept 7, 2019 & County park\tabularnewline
Durga Puja & October 4-6, 2019 & Preferably Gaithersburg HS or Northwest
HS\tabularnewline
\bottomrule
\end{longtable}

\hypertarget{transfers-from-bod-18}{%
\subsection{Transfers from BOD 18}\label{transfers-from-bod-18}}

\hypertarget{action-items-5}{%
\paragraph{Action items}\label{action-items-5}}

\begin{enumerate}
\def\labelenumi{\arabic{enumi}.}
\tightlist
\item
  Ms.~Bera will receive the storage key from Mr.~Ghosh via Ms.~Nag
\item
  Mr.~Dasgupta will update all e-mail accounts to reflect new membership
\item
  Mr.~Dasgupta will work with the new members in the onboarding process
\item
  Mr.~Dasgupta will add Ms.~Bera to both the
  \href{mailto:president@sanskriti-dc.org}{\nolinkurl{president@sanskriti-dc.org}}
  and
  \href{mailto:culsec@sanskriti-dc.org}{\nolinkurl{culsec@sanskriti-dc.org}}
  accounts
\end{enumerate}


\end{document}
